\documentclass[a4paper,12pt]{article}
\usepackage{zh_CN-Adobefonts_external} % Simplified Chinese Support using external fonts (./fonts/zh_CN-Adobe/)
\usepackage{fancyhdr}  % 页眉页脚
\usepackage{minted} % 代码高亮
\usepackage[colorlinks]{hyperref}  % 目录可跳转
\setlength{\headheight}{15pt}
\hypersetup{colorlinks=false}

% 定义页眉页脚
\pagestyle{fancy}
\fancyhf{}
\fancyhead[C]{Algorithm Template}
\renewcommand{\headrulewidth}{1pt}
\renewcommand{\footrulewidth}{1pt}
% \lfoot{}
% \fancyfoot[L]{© Xiaobin Ren}
\fancyfoot[C]{\thepage}
% \rfoot{\thepage}

\author{Xiaobin Ren @ BUPT}
\title{Algorithm For Competitive Programming}

\begin{document}
\maketitle % 封面
\thispagestyle{empty} %封面不显示页码
\newpage % 换页
\tableofcontents % 目录

%模板
% \newpage
% \section{}
% \subsection{}
% \inputminted[breaklines, linenos]{c++}{/.cc}

\newpage
\section{基础算法}
\subsection{排序}
\inputminted[breaklines, linenos]{c++}{basic/sort.cc}
\subsection{二分}
\inputminted[breaklines, linenos]{c++}{basic/binary_search.cc}
\subsection{高精度}
\inputminted[breaklines, linenos]{c++}{basic/high_precision.cc}
\subsection{前缀和与差分}
\inputminted[breaklines, linenos]{c++}{basic/prefixsum_diff.cc}
\subsection{双指针算法}
\inputminted[breaklines, linenos]{c++}{basic/double_pointer.cc}
\subsection{位运算}
\inputminted[breaklines, linenos]{c++}{basic/bit.cc}
\subsection{离散化}
\inputminted[breaklines, linenos]{c++}{basic/disc.cc}
\subsection{区间合并}
\inputminted[breaklines, linenos]{c++}{basic/interval_union.cc}
\subsection{启发式合并}
\inputminted[breaklines, linenos]{c++}{basic/heuristic_merge.cc}
\subsection{RMQ一维/二维}
\inputminted[breaklines, linenos]{c++}{basic/rmq.cc}
\subsection{CDQ分治}
\inputminted[breaklines, linenos]{c++}{basic/cdq.cc}
\subsection{C++ STL}
\inputminted[breaklines, linenos]{md}{basic/stl.cc}

\newpage
\section{数据结构}
\subsection{链表}
\inputminted[breaklines, linenos]{c++}{ds/list.cc}
\subsection{栈/单调栈}
\inputminted[breaklines, linenos]{c++}{ds/stack.cc}
\subsection{队列/单调队列}
\inputminted[breaklines, linenos]{c++}{ds/queue.cc}
\subsection{Trie}
\inputminted[breaklines, linenos]{c++}{ds/trie.cc}
\subsection{并查集}
\inputminted[breaklines, linenos]{c++}{ds/dsu.cc}
\subsection{手写哈希表}
\inputminted[breaklines, linenos]{c++}{ds/hash.cc}
\subsection{树状数组}
\inputminted[breaklines, linenos]{c++}{ds/tree_array.cc}
\subsection{线段树}
\inputminted[breaklines, linenos]{c++}{ds/seg_tree.cc}
\subsection{可持久化数据结构}
\subsubsection{可持久化Trie}
\inputminted[breaklines, linenos]{c++}{ds/last_trie.cc}
\subsubsection{可持久化线段树}
\inputminted[breaklines, linenos]{c++}{ds/last_seg.cc}
\subsection{平衡树}
\subsubsection{Treap}
\inputminted[breaklines, linenos]{c++}{ds/treap.cc}
\subsubsection{Splay}
\inputminted[breaklines, linenos]{c++}{ds/splay.cc}
\subsection{树套树}
\subsubsection{线段树套STL}
\inputminted[breaklines, linenos]{c++}{ds/stl_in_seg.cc}
\subsubsection{线段树套Splay}
\inputminted[breaklines, linenos]{c++}{ds/splay_in_seg.cc}
\subsubsection{线段树套线段树}
\inputminted[breaklines, linenos]{c++}{ds/seg_in_seg.cc}
\subsection{分块}
\inputminted[breaklines, linenos]{c++}{ds/decompose.cc}
\subsection{树链剖分}
\inputminted[breaklines, linenos]{c++}{ds/hld.cc}
\subsection{树分治}
\subsubsection{点分治}
\inputminted[breaklines, linenos]{c++}{ds/tree_divide.cc}
\subsection{动态树}
\subsubsection{Link Cut Tree}
\inputminted[breaklines, linenos]{c++}{ds/lct.cc}
\subsection{左偏树}
\inputminted[breaklines, linenos]{c++}{ds/leftist.cc}
\subsection{圆方树}
\inputminted[breaklines, linenos]{c++}{ds/cactus.cc}
\subsection{Dancing Links}
\subsubsection{精确覆盖问题}
\inputminted[breaklines, linenos]{c++}{ds/exact_cover.cc}
\subsubsection{重复覆盖问题}
\inputminted[breaklines, linenos]{c++}{ds/multi_cover.cc}
\subsection{莫队算法}
\subsubsection{朴素莫队}
\inputminted[breaklines, linenos]{c++}{ds/mo.cc}
\subsubsection{带修改莫队}
\inputminted[breaklines, linenos]{c++}{ds/modify_mo.cc}
\subsubsection{回滚莫队}
\inputminted[breaklines, linenos]{c++}{ds/rollback_mo.cc}
\subsubsection{树上莫队}
\inputminted[breaklines, linenos]{c++}{ds/tree_mo.cc}
\subsubsection{二次离线莫队}
\inputminted[breaklines, linenos]{c++}{ds/offline_mo.cc}


\newpage  
\section{搜索}
\subsection{BFS}
\subsubsection{Flood Fill}
\inputminted[breaklines, linenos]{c++}{search/bfs/flood.cc}
\subsubsection{数码问题}
\inputminted[breaklines, linenos]{c++}{search/bfs/shuma.cc}
\subsubsection{BFS最短路}
\inputminted[breaklines, linenos]{c++}{search/bfs/path.cc}
\subsubsection{最少步数模型}
\inputminted[breaklines, linenos]{c++}{search/bfs/min_step.cc}
\subsubsection{多源BFS}
\inputminted[breaklines, linenos]{c++}{search/bfs/mul_source.cc}
\subsubsection{双端队列BFS}
\inputminted[breaklines, linenos]{c++}{search/bfs/deq_bfs.cc}
\subsubsection{双向BFS}
\inputminted[breaklines, linenos]{c++}{search/bfs/2side_bfs.cc}
\subsubsection{A*}
\inputminted[breaklines, linenos]{c++}{search/bfs/astar.cc}
\subsection{DFS}
\subsubsection{DFS连通性}
\inputminted[breaklines, linenos]{c++}{search/dfs/link.cc}
\subsubsection{迭代加深搜索}
\inputminted[breaklines, linenos]{c++}{search/dfs/diedai.cc}
\subsubsection{双向DFS}
\inputminted[breaklines, linenos]{c++}{search/dfs/2side.cc}
\subsubsection{IDA*}
\inputminted[breaklines, linenos]{c++}{search/dfs/ida.cc}
\subsection{模拟退火}
\inputminted[breaklines, linenos]{c++}{search/simulate_anneal.cc}


\newpage
\section{图论} % 一级标题
\subsection{图的BFS/DFS}  % 1
\subsubsection{树的重心}
\inputminted[breaklines, linenos]{c++}{graph/bdfs/grav.cc}
\subsubsection{BFS最短路/点的层次}
\inputminted[breaklines, linenos]{c++}{graph/bdfs/cenci.cc}
\subsubsection{BFS最短路计数}
\inputminted[breaklines, linenos]{c++}{graph/bdfs/sum.cc}
\subsection{最短路} % 2
\subsubsection{SPFA}
\inputminted[breaklines, linenos]{c++}{graph/shortest_path/spfa.cc}
\subsubsection{Bellman-Ford}
\inputminted[breaklines, linenos]{c++}{graph/shortest_path/bellman.cc}
\subsubsection{Dijkstra}
\inputminted[breaklines, linenos]{c++}{graph/shortest_path/dij.cc}
\subsubsection{Floyd}
\inputminted[breaklines, linenos]{c++}{graph/shortest_path/floyd.cc}
\subsection{最小生成树} %3
\subsubsection{Kruskal} 
\inputminted[breaklines, linenos]{c++}{graph/mst/kruskal.cc}
\subsubsection{Prim} 
\inputminted[breaklines, linenos]{c++}{graph/mst/prim.cc}
\subsection{二分图} % 4
\subsubsection{染色法} 
\inputminted[breaklines, linenos]{c++}{graph/2graph/judge.cc}
\subsubsection{匈牙利算法} 
\inputminted[breaklines, linenos]{c++}{graph/2graph/xyl.cc}
\subsection{差分约束} % 5
\inputminted[breaklines, linenos]{c++}{graph/cfys/cfys.cc}
\subsection{最近公共祖先} % 6
\inputminted[breaklines, linenos]{c++}{graph/lca/lca.cc}
\subsection{Tarjan算法} % 7
\subsubsection{有向图强连通分量} 
\inputminted[breaklines, linenos]{c++}{graph/tarjan/scc.cc}
\subsubsection{无向图边双连通分量} 
\inputminted[breaklines, linenos]{c++}{graph/tarjan/edcc.cc}
\subsubsection{无向图点双连通分量} 
\inputminted[breaklines, linenos]{c++}{graph/tarjan/vdcc.cc}
\subsection{欧拉回路/路径} % 8
\inputminted[breaklines, linenos]{c++}{graph/eluer/eluer.cc}
\subsection{拓扑排序} % 9
\inputminted[breaklines, linenos]{c++}{graph/topsort/top.cc}
\subsection{最大流}
\subsubsection{Dinic模板}
\inputminted[breaklines, linenos]{c++}{graph/flow/dinic.cc}
\subsubsection{二分图最优匹配}
\inputminted[breaklines, linenos]{c++}{graph/flow/match.cc}
\subsubsection{无源汇上下界可行流}
\inputminted[breaklines, linenos]{c++}{graph/flow/no_st_flow.cc}
\subsubsection{有源汇上下界最大流}
\inputminted[breaklines, linenos]{c++}{graph/flow/st_maxflow.cc}

\newpage
\section{数学}
\subsection{质数/筛法}
\inputminted[breaklines, linenos]{c++}{math/prime.cc}
\subsection{约数}
\inputminted[breaklines, linenos]{c++}{math/yueshu.cc}
\subsection{欧拉函数}
\inputminted[breaklines, linenos]{c++}{math/eluer.cc}
\subsection{快速幂/龟速乘}
\inputminted[breaklines, linenos]{c++}{math/qmi.cc}
\subsection{扩展欧几里得算法}
\inputminted[breaklines, linenos]{c++}{math/exgcd.cc}
\subsection{中国剩余定理}
\inputminted[breaklines, linenos]{c++}{math/crt.cc}
\subsection{高斯消元}
\inputminted[breaklines, linenos]{c++}{math/gauss.cc}
\subsection{组合数}
\inputminted[breaklines, linenos]{c++}{math/combine.cc}
\subsection{莫比乌斯函数}
\inputminted[breaklines, linenos]{c++}{math/mobius.cc}
\subsection{矩阵乘法}
\inputminted[breaklines, linenos]{c++}{math/matrix.cc}
\subsection{容斥原理}
\inputminted[breaklines, linenos]{c++}{math/rongchi.cc}
\subsection{概率期望}
\inputminted[breaklines, linenos]{c++}{math/qiwang.cc}
\subsection{博弈论}
\inputminted[breaklines, linenos]{c++}{math/game.cc}

\newpage
\section{动态规划}
\subsection{背包}
\subsubsection{01背包}
\paragraph{朴素01背包}
\inputminted[breaklines, linenos]{c++}{dp/bag/01.cc}
\paragraph{01背包求方案数}
\inputminted[breaklines, linenos]{c++}{dp/bag/sum.cc}
\paragraph{01背包求具体方案}
\inputminted[breaklines, linenos]{c++}{dp/bag/01_way.cc}
\subsubsection{完全背包}
\inputminted[breaklines, linenos]{c++}{dp/bag/comp.cc}
\subsubsection{多重背包}
\inputminted[breaklines, linenos]{c++}{dp/bag/mul.cc}
\subsubsection{分组背包}
\inputminted[breaklines, linenos]{c++}{dp/bag/group.cc}
\subsubsection{混合背包}
\inputminted[breaklines, linenos]{c++}{dp/bag/mix.cc}
\subsubsection{二维费用背包}
\inputminted[breaklines, linenos]{c++}{dp/bag/2di.cc}
\subsubsection{有依赖的背包}
\inputminted[breaklines, linenos]{c++}{dp/bag/dep.cc}

\subsection{线性DP}
\subsubsection{LIS/Dilworth定理}
\inputminted[breaklines, linenos]{c++}{dp/linear/lis.cc}
\subsubsection{LCS}
\inputminted[breaklines, linenos]{c++}{dp/linear/lcs.cc}
\subsection{区间DP}
\inputminted[breaklines, linenos]{c++}{dp/interval.cc}
\subsection{计数DP}
\inputminted[breaklines, linenos]{c++}{dp/jishu.cc}
\subsection{数位DP}
\inputminted[breaklines, linenos]{c++}{dp/shuwei.cc}
\subsection{状压DP}
\inputminted[breaklines, linenos]{c++}{dp/zhuangya.cc}
\subsection{树形DP}
\inputminted[breaklines, linenos]{c++}{dp/treedp.cc}
\subsection{单调队列优化DP}
\inputminted[breaklines, linenos]{c++}{dp/dddl.cc}
\subsection{斜率优化DP}
\inputminted[breaklines, linenos]{c++}{dp/xielv.cc}


\newpage
\section{字符串}
\subsection{KMP}
\inputminted[breaklines, linenos]{c++}{string/kmp.cc}
\subsection{AC自动机}
\inputminted[breaklines, linenos]{c++}{string/ac_auto.cc}
\subsection{字符串哈希}
\inputminted[breaklines, linenos]{c++}{string/hash.cc}
\subsection{最小表示法}
\inputminted[breaklines, linenos]{c++}{string/minimal.cc}
\subsection{Manacher算法}
\inputminted[breaklines, linenos]{c++}{string/manacher.cc}
\subsection{后缀数组}
\inputminted[breaklines, linenos]{c++}{string/sa.cc}
\subsection{后缀自动机}
\inputminted[breaklines, linenos]{c++}{string/sam.cc}


\newpage
\section{比赛现场配置}
\subsection{VIM现场赛配置}
\inputminted[breaklines, linenos]{bash}{others/vim.bash}
\subsection{Snippet代码头文件}
\inputminted[breaklines, linenos]{c++}{others/head.cc}


\end{document}